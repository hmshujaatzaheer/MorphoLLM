% ============================================================================
% LATEX CODE TO ADD REPOSITORY BOX TO PHD PROPOSAL
% ============================================================================
%
% This file contains the LaTeX code to add a professional repository
% information box to your phd_proposal.tex file.
%
% ============================================================================

% ============================================================================
% STEP 1: Add this package to the preamble (if not already present)
% ============================================================================
% Location: After the existing \usepackage declarations (around line 20)
% ----------------------------------------------------------------------------

\usepackage{tcolorbox}
\tcbuselibrary{skins,breakable}

% ============================================================================
% STEP 2: Add the repository box definition to the preamble
% ============================================================================
% Location: After the color definitions (around line 35, after \definecolor)
% ----------------------------------------------------------------------------

% Repository box style
\newtcolorbox{repobox}{
    enhanced,
    colback=blue!5!white,
    colframe=deepblue,
    fonttitle=\bfseries\large,
    title={\faGithub\ Open Source Implementation},
    attach boxed title to top left={yshift=-2mm, xshift=5mm},
    boxed title style={colback=deepblue, colframe=deepblue},
    left=5mm,
    right=5mm,
    top=4mm,
    bottom=4mm,
    breakable
}

% If you want to use Font Awesome icon, add this package:
% \usepackage{fontawesome5}
% Otherwise, remove \faGithub\ from the title above

% ============================================================================
% STEP 3: Add the repository section to the document
% ============================================================================
% RECOMMENDED LOCATION: After Section 7 (Validity and Feasibility), 
%                       before Section 8 (Conclusion)
%
% Alternative locations:
% - After the Conclusion section
% - In Section 6 (What Is Invented and Its Value)
% ----------------------------------------------------------------------------

% Add after line ~710 (after the timeline figure, before \section{Conclusion})
% Or search for "\section{Conclusion}" and add just before it

\section{Open Source Implementation}
\label{sec:implementation}

The complete implementation of the MorphoLLM framework is available as an 
open-source Python package, enabling reproducibility and community contribution.

\begin{repobox}
\textbf{Repository:} \url{https://github.com/hmshujaatzaheer/MorphoLLM}

\vspace{2mm}
\textbf{Package Contents:}
\begin{itemize}[leftmargin=*, nosep]
    \item \texttt{morphollm/core/} -- Extended Configuration Space $\mathcal{C}_{ext}$, 
          Morphology-Augmented Dynamics, Stability Analysis
    \item \texttt{morphollm/algorithms/} -- L2MT and SM-MPC implementations 
          with full Algorithm 1 and 2
    \item \texttt{morphollm/models/} -- LLM interface supporting GPT-4, Claude, 
          and local models
    \item \texttt{examples/} -- Demonstration scripts including full pipeline demo
    \item \texttt{tests/} -- Comprehensive test suite with >80\% coverage
\end{itemize}

\vspace{2mm}
\textbf{Key Features:}
\begin{itemize}[leftmargin=*, nosep]
    \item PyTorch-based differentiable implementations for end-to-end learning
    \item Modular design allowing custom LLM backends and physics engines
    \item Real-time capable SM-MPC with warm-starting optimization
    \item Stability-guaranteed morphology adaptation via Theorem 1
\end{itemize}

\vspace{2mm}
\textbf{Installation:}
\begin{verbatim}
pip install morphollm
# or from source:
git clone https://github.com/hmshujaatzaheer/MorphoLLM.git
cd MorphoLLM && pip install -e ".[all]"
\end{verbatim}

\vspace{2mm}
\textbf{Quick Start:}
\begin{verbatim}
from morphollm import L2MT, MorphologySpace
l2mt = L2MT(morphology_space=MorphologySpace(dim=12))
trajectory = l2mt.generate("Pick up the fragile vase carefully")
\end{verbatim}
\end{repobox}

The repository follows software engineering best practices including 
continuous integration, comprehensive documentation, and semantic versioning.
All novel contributions (Extended Configuration Space, L2MT, SM-MPC, 
Morphological Stability Theorem) are implemented with detailed inline 
documentation mapping to the theoretical formulations presented in this proposal.

% ============================================================================
% ALTERNATIVE: Simpler version without tcolorbox
% ============================================================================
% If you prefer not to add tcolorbox, use this simpler version:

\section{Open Source Implementation}
\label{sec:implementation}

\begin{center}
\fbox{\parbox{0.9\textwidth}{
\centering
\textbf{\large Open Source Implementation Available}\\[2mm]
\textbf{Repository:} \url{https://github.com/hmshujaatzaheer/MorphoLLM}\\[2mm]
\textbf{License:} MIT | \textbf{Language:} Python 3.9+ | \textbf{Framework:} PyTorch\\[2mm]
Complete implementation of L2MT Algorithm, SM-MPC Controller, and Stability Analysis
}}
\end{center}

The MorphoLLM framework is released as an open-source Python package containing:
\begin{itemize}
    \item Extended Configuration Space formalism with PyTorch-differentiable operations
    \item L2MT algorithm with modular LLM backends (GPT-4, Claude, local models)
    \item SM-MPC controller with stability-guaranteed morphology adaptation
    \item Comprehensive examples and documentation for reproducibility
\end{itemize}

% ============================================================================
% WHERE TO INSERT IN phd_proposal.tex
% ============================================================================
%
% OPTION A (Recommended): Insert as new section before Conclusion
% 
% Find this line (around line 710-720):
%   \section{Conclusion}
%
% Insert the repository section code BEFORE this line.
%
% ----------------------------------------------------------------------------
%
% OPTION B: Insert within Section 6 (What Is Invented)
%
% Find this line (around line 590-600):
%   \subsection{Strategic Alignment with CREATE Lab}
%
% Insert after this subsection, before the section ends.
%
% ----------------------------------------------------------------------------
%
% OPTION C: Insert after Conclusion as Appendix
%
% Find the References section:
%   \begin{thebibliography}
%
% Insert before this, after Conclusion.
%
% ============================================================================

% ============================================================================
% COMPLETE INSERTION EXAMPLE
% ============================================================================
% Here's exactly what your document should look like around the insertion point:
%
% ... [end of Section 7 content] ...
%
% \section{Open Source Implementation}    % <-- ADD THIS
% \label{sec:implementation}              % <-- ADD THIS
%                                         
% [Repository box content here]           % <-- ADD THIS
%
% \section{Conclusion}                    % <-- EXISTING
% \label{sec:conclusion}                  % <-- EXISTING
%
% ============================================================================
